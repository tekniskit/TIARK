%!TEX root = ../Master.tex

\section{Introduction} % (fold)
\label{sec:theory_introduction}

This report is written as part of the course 'Artificial Intelligence in Robotics'. The sections in this chapter presents the knowledge of the theory gained througout the course. The curriculum consists primary of video lectures from the course 'Artificial Intelligence for Robotics' at udacity.com, supplemented with selected sections from the textbook 'Probabilistic Robotics' by Sebastian Thrun, Wolfram Burgard and Dieter Fox.\\

One of the main subjects of this course is localization which deals with the problem of locating a robot in some environment using methods such as Markov Localization, Kalman Filter, and Particle Filter. When the robots location is known, the next step is planning a path for the robot to follow. To solve this problem the course teaches the algorithm A*, which is an extension of Dijkstra's shortest path algorithm. Robot Motion is treated in the course by looking at methods for generating smooth paths and controlling the robot using PID Control.\\

Lastly the subject of Simultaneous Localization And Mapping (SLAM) is introduced. Until now, a map of the robots environment is given. But, as the name suggest, this methods localises the robot and generates a map of the environment at the same time. Hence this is a somewhat advanced subject, however, the course limits its treatment of this subjects to a variant of SLAM called GraphSLAM.

% section introduction (end)
