%!TEX root = ../Master.tex

\section{Hardware} % (fold)
\label{sec:hardware}

As mentioned in \autoref{sec:project_introduction} the focus of the project have been on the software implementations of theory learned throughout the course.
By using the Neato XV-11 vacuum cleaning robot (shown in \autoref{fig:neato_xv11}) which have a build-in lidar to sense its environment, and that implements functionality to move by some unit distance, most hardware concerns have been eliminated.\\

The lidar has a resolution of 360 measurements per revolution and can measure up to 5 meters.\\

The Neato robot has a serial interface accessable through a USB port on the back of the robot. This means that there is no support for wirelessly communicating with the robot and we chose not to implement it in the project. Trough the serial interface the lidar scan can be read and motion commands can be sent.

\begin{figure}[H]
\centering
\includegraphics[scale=0.40]{images/neato_xv11}
\caption{Neato XV-11}
\label{fig:neato_xv11}
\end{figure}

% section hardware (end)
