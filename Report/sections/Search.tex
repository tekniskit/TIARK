%!TEX root = ../Master.tex

\section{Path Planning} % (fold)
\label{sec:path_planning}

When a robot has located itself the next task might be to go somewhere. The task of planning a path from the location of the robot to some goal location is, when talking about robots, often referred to as robot motion planning. The problem to solve is to find the best possible path for the goal location when knowing the start location, the map and maybe some kind of cost function. This cost function can be thought of as the time it takes to follow a certain route. When introducing this cost function the best possible path is the minimum cost path. \\

\subsection{A* Algorithm} % (fold)
\label{sub:a_algorithm}

The path planning can be looked at as a search problem. A search problem is a problem with a set of nodes, an initial node and a goal node. The search problem can be solved by using a search algorithm. The search algorithm starts at the initial node and searches throughout the nodes in order to find the goal node. A variant of this search algorithm has been developed in order to handle the problem of path planning better. This algorithm is referred to as A* (pronounced A star). The steps of the algorithm is explained in the following section. \\

To understand the approach of the A* algorithm a couple of variables must be introduced.

\begin{itemize}
	\item The \emph{g} value is the cost of the path used to get to that node.

	\item The \emph{h} value is the number of nodes to pass to get to the goal node without taking obstacles into account. This can also be looked at as the displacement along both the x-axis and the y-axis of a node compared to the goal node.

	\item The \emph{f} value is the sum of the \emph{g} value and the \emph{h} value.\\
\end{itemize}

The algorithm also introduces three different lists.

\begin{itemize}
	\item The \emph{open} list is a list of nodes to be investigated. The list is sorted so that the node with the smallest \emph{f} value is the next node to be investigated. Nodes are being added to the \emph{open} list when a neighbor node is being investigated.

	\item The \emph{closed} list is a list of all nodes which have been investigated. This is used to keep track of these nodes in order to avoid investigation of a node twice or more.

	\item The \emph{action} list is a list which keeps track of which node was the previous node for all investigated nodes.\\
\end{itemize}

The procedure of the algorithm is as following. 

\begin{enumerate}
  \item The initial node is added to the \emph{open} list.

  \item The node in the \emph{open} list with the lowest \emph{f} value is chosen to be investigated. The chosen node is deleted from the \emph{open} list and added to the \emph{closed} list. 

  \item All neighbor nodes which are not in the \emph{closed} list are added to the \emph{open} list. All nodes are added to the \emph{action} list together with the information of which node was the previous node.
\end{enumerate}

Step two and three are repeated until the node from the \emph{open} list is the goal node. Then the planned path can be found. This is done by introducing a new list called \emph{path}. The goal node is added to this list and from the \emph{action} list the previous node is known. This node is added to the \emph{path} list. The previous node to this node is looked up in the \emph{action} list and so on until the previous node is the initial node. After reversing the \emph{path} list the path for the robot is planned. \\

% subsection a_algorithm (end)

\subsection{Path Smoothing} % (fold)
\label{sub:path_smoothing}

As described the A* algorithm results in a \emph{path} list. This planned path might be a very cornered path with lots of sharp turns for the robot to do. If this is not desirable for the robot the path can be smoothed. \\

For the smoothing algorithm all nodes in the \emph{path} list are denoted $x_i$ where $i$ is the number the node has in the list. All nodes in the \emph{smoothened path} list is denoted $y_i$. The algorithm has to solve a minimization problem. The \emph{smoothened path} list is initialized to be the same as the \emph{path} list as it is denoted in \autoref{eq:init_smooth_path}.

\begin{equation}
\label{eq:init_smooth_path}
y_i = x_i
\end{equation}

The minimization problem is defined in \autoref{eq:minimize}. The factor $\alpha$ sets how smooth the path is going to be. The larger value of $\alpha$, the more smooth the path gets. 

\begin{equation}
\label{eq:minimize}
\text{minimize\ } (x_i - y_i)^2 - \alpha (y_i - y_{i+1})^2
\end{equation}

In an implementation of the path smoothening algorithm two values are used to regulate for the smoothening, $\alpha$ and $\beta$. The $\alpha$ value yields how much the path should smoothen and the $\beta$ value yields how much the path should look like the original path. The path is being changed until the new change is smaller than some tolerance. An example of an implementation is shown in \autoref{smooth_path}. Notice that when the algorithm is implemented the initial node and the goal node has to stay unchanged.

\begin{lstlisting}[caption=Implementation example of path smoothing., label=smooth_path, language=Matlab]
function newpath = smoothPath(path, alpha, beta, tolerance)
    [length, width] = size(path);
    newpath = zeros(length,width);
    
    for i = 1:length
        for j = 1:width
            newpath(i,j) = path(i,j);
        end
    end
    
    change = tolerance;
    while change >= tolerance
        change = 0;
        for n = 2:length - 1
            for m = 1:width
                aux = newpath(n,m);
                newpath(n,m) = newpath(n,m) + alpha * (path(n,m) - newpath(n,m));
                newpath(n,m) = newpath(n,m) + beta * (newpath(n-1,m)+newpath(n+1,m)-2*newpath(n,m));
                change = change + abs(aux - newpath(n,m));
            end
        end
    end
end
\end{lstlisting}

% subsection path_smoothing (end)
% section path_planning (end)