%!TEX root = ../Master.tex

\section{Software} 
\label{sec:software}

Overall architecture.
Use of ROS and RVIZ.
Why didn't we use the navigation stack and other ros packages?

\subsection{ROS}
To be able to easily test and validate the implemented AI methods, we used ROS to visualize the system.

ROS (Robot Operating System) is a collection of software for building robotsystems.
ROS has a large collection of packages for 3D visualization, navigation, mapping, localization, planning, etc.  

Only the visualization and communication parts of ROS has been used in this project.
This was done because we wanted to write our own implementation of localization, planning and control.

A ROS project consists of a collection of processes with a single responsability, i.e. planning, vision, control, etc. 
ROS also handles the communication between these processes.\\
The processes are conceptualized as nodes in a network which can communication by publishing and subscriping on topics.

The communication in ROS consists of messages with a specific data type.
The datatype could be anything from simple integers to LaserScans which descripes a scan from a lidar and extra header information as a timestamp and a sequence number if the order of scans are important.

A node can also contain services which can be called with a request-response communication model, instead of the the publish-subscripe model between nodes.

ROS master is the program that coordinates the communication between the nodes.
It also works as a registry for nodes, services, topics, etc.

\subsection{ROS Nodes}
\subsection{Map}
\subsection{Lidar}
\subsection{Particle}
\subsection{Monte Carlo Localizer}
\subsection{Explorer}
\subsection{PathFollower}
\subsection{AStarPlanner}
