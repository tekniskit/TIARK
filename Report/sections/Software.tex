%!TEX root = ../Master.tex

\section{Software} 
\label{sec:software}

Overall architecture.
Use of ROS and RVIZ.
Why didn't we use the navigation stack and other ros packages?
We used c++ and the following sections, except ROS, will descripe each class we wrote.

\subsection{ROS}
To be able to easily test and validate the implemented AI methods, we used ROS to visualize the system.

ROS (Robot Operating System) is a collection of software for building robotsystems.
ROS has a large collection of packages for 3D visualization, navigation, mapping, localization, planning, etc.  

Only the visualization and communication parts of ROS has been used in this project.
This was done because we wanted to write our own implementation of localization, planning and control.

A ROS project consists of a collection of processes with a single responsability, i.e. planning, vision, control, etc. 
ROS also handles the communication between these processes.\\
The processes are conceptualized as nodes in a network which can communication by publishing and subscriping on topics.

The communication in ROS consists of messages with a specific data type.
The datatype could be anything from simple integers to LaserScans which descripes a scan from a lidar and extra header information as a timestamp and a sequence number if the order of scans are important.

A node can also contain services which can be called with a request-response communication model, instead of the the publish-subscripe model between nodes.

ROS master is the program that coordinates the communication between the nodes.
It also works as a registry for nodes, services, topics, etc.

\subsection{ROS Nodes}
The system implements two ROS nodes, the first is a driver for Neato robot and the second is the navigation system.\\

The driver is a modified version of a open source driver for the robot. \fixme{link to github}
The open source driver implements an older version of the serial protocol for the Neato robot.
This have been modified to implement the newer protocol which our robot uses.\\

The navigation node implements all of the AI algoritms; localization, planning and control.
The two nodes communicate using ROS messages for lidar scans and control commands. 

\subsection{Map}
The map is implemented as a matrix of grid cells where each grid cell represents a 25cm*25cm square in the real world.
Each grid cell consists of an integer describing what is in the grid cell. For instance every cell descriping a wall has the value 100, open floor is 20, the preferred path of the robot is 1.
These specific values are choosen to make the path planning easier as they descripe the cost of driving in that specific area.
The map is also used when simulating lidar scans.

\subsection{Lidar}
The localization require the ability to simulate a lidar scan in the given map.
This is implemented using a ray casting algorithm.
Since this was not part of the course corriculum, an open source implementation has been used.\\

It can calculate the distance from a given point to the nearest wall in a given direction.

\subsection{Monte Carlo Localizer}
The localization in the project is implemented using the Monte Carlo localization algorithm.

Initially all the particles are randomly spread uniformly across the map.
Each time we get a lidar scan from the robot we update each particles weight as the average between the existing weight and the new weight.

The weights are calculated using the lidar scan from the robot and a simulated scan from the Lidar class using the particles pose.
A guassian is used to calculate the weight of a measurement:
$$x_i = \frac{p_i - r_i}{r_i}$$
$$w = \sum_{i=0}^N exp\left(-\frac{{x_i}^2}{2\sigma^2}\right)$$

\subsection{Explorer}
\subsection{PathFollower}
\subsection{AStarPlanner}
